\documentclass[11pt,xcolor=dvipsnames]{beamer}
\usetheme{Warsaw}
\usepackage[utf8]{inputenc}
\usepackage[french]{babel}
\usepackage[T1]{fontenc}
\usepackage{tikz}
\usetikzlibrary{calc,fit}
\usepackage{amsmath}
\usepackage{amsfonts}
\usepackage{amssymb}
\usepackage{dirtree}
\setbeamercolor{frametitle}{bg=orange}
\setbeamercolor{framenumber}{bg=black,fg=white}
\setbeamerfont{page number in head/foot}{size=\large}
\setbeamercolor{structure}{fg=orange!85!black}
\setbeamertemplate{footline} {
  \leavevmode%
  \hbox{%
  \begin{beamercolorbox}[wd=.5\paperwidth,ht=2.25ex,dp=1ex,center]{author in head/foot}%
    \usebeamerfont{author in head/foot}\insertshortauthor
  \end{beamercolorbox}%
  \begin{beamercolorbox}[wd=.4\paperwidth,ht=2.25ex,dp=1ex,center]{title in head/foot}%
    \usebeamerfont{title in head/foot}\insertshorttitle\hspace*{3em}
  \end{beamercolorbox}}%
  \begin{beamercolorbox}[wd=.1\paperwidth,ht=2.25ex,dp=1ex,center]{framenumber}
     \usebeamerfont{framenumber}\insertframenumber{} / \inserttotalframenumber\hspace*{1ex}
  \end{beamercolorbox}%
  \vskip0pt%
}

\pgfdeclareimage[height=0.6cm]{logoUPMC}{UPMC.png}
\logo{\pgfuseimage{logoUPMC}}

\author{Corentin GUILLEVIC et Sarah HADBI}
\title{Projet Paquito}
%\setbeamercovered{transparent} 
%\setbeamertemplate{navigation symbols}{} 
%\logo{} 
%\institute{} 
\date{24 avril 2015} 
%\subject{} 
\begin{document}
\begin{frame}
\titlepage
\end{frame}

\section{Le projet Paquito}
\subsection{Besoin}
\begin{frame}{Le besoin}
\begin{tikzpicture}[scale=0.8]
	\node[] (U) at (1,-3) {\includegraphics[width=30pt]{../livrables/img/programmer.png}} ;
	\node[] (G) at (1,-1) {\includegraphics[width=25pt]{../livrables/img/new/server.png}} ;
	\node[] (N) at (5,-2) {\includegraphics[width=130pt]{../livrables/img/new/cloud.png}} ;	

	% Bases de données
	\foreach \pos [count=\i] in {{(9.5,0)}, {(9.5,-2)}, {(9.5,-4)}}
		\node[] (M\i) at \pos {\includegraphics[width=40pt]{../livrables/img/new/database.png}} ;
	% Icones des distributions
	\foreach \pos/\size/\image in {{(9.5,0)}/30/debian_logo, {(9.5,-2)}/25/redhat_logo, {(9.5,-4)}/25/Arch-linux-logo}
		\node[] () at \pos {\includegraphics[width=\size pt]{../livrables/img/\image.png}} ;
	% Ordinateurs des clients
	\foreach \pos [count=\i] in {{(12.3,-4)}, {(12.3,-2)}, {(12.3,0)}}
		\node[] (C\i) at \pos {\includegraphics[width=35pt]{../livrables/img/new/computer.png}} ;
	% Processeurs (architecture)
	\foreach \pos/\j in {{(11.4,-0.8)}/32, {(13.2,-0.8)}/64, {(11.4,-2.8)}/32, {(13.2,-2.8)}/64}
			\node[] at \pos {\includegraphics[width=17pt]{../livrables/img/new/processor\j.png}} ;
	\node[] at (13.3,-4.6) {\colorbox{white}{\makebox(12,12){\includegraphics[width=17pt]{../livrables/img/new/processor64.png}}}} ;
	% Utilisateurs
	\foreach \i in {{(13.5,0.1)}, {(13.5,-2.0)}, {(13.5,-3.8)}}
		\node[] at \i {\includegraphics[width=15pt]{../livrables/img/latex_user.png}} ;

	% Textes
	\node at (U.south) [below] {Développeur} ;
	\node at (G.north) [above] {GitHub} ;
	\node at (M1.north) [above] {Miroirs (dépôts)} ;
	\node at (C1.south) [below] {Clients} ;
	\node at (N.center) [] {\textbf{{\Huge ?}}} ;	

	% Flèches unidirectionnelles
	\foreach \a/\b in {U/G, G/N, N/M1, N/M2, N/M3}
		\draw[-stealth,very thick] (\a) -- (\b);
	% Flèches bidirectionnelles
	\foreach \a/\b in {M1/C3, M2/C2, M3/C1}
		\draw[stealth-stealth,very thick] (\a) -- (\b) ;
\end{tikzpicture}
\end{frame}

\subsection{Solution}
\begin{frame}{La solution}
\begin{tikzpicture}[scale=0.7]
	\node[] (U) at (-4,-4) {\includegraphics[width=30pt]{../livrables/img/programmer.png}} ;
	\node[] (G) at (-4,-1) {\includegraphics[width=25pt]{../livrables/img/new/server.png}}
	node at (G.north) [above] {GitHub} ;
	\node[anchor=south west] at (-3.3,-1.2) {\includegraphics[width=12pt]{../livrables/img/new/flag.png}} ;
	\node[] (J) at (-1.8,-2) {\includegraphics[width=25pt]{../livrables/img/new/server.png}}
	node at (J.south) [below] {Jenkins CI} ;
	\draw[-stealth,very thick] (G) -- (J);
	\draw[-stealth,very thick] (U) -- (G);

	\foreach \op/\label/\pos in {1.0/D1/{(1.5,0)}, 0.5/D2/{(1.5,-2)}, 0.2/D3/{(1.5,-4)}} {
		\node[opacity=\op] (\label) at \pos {\includegraphics[width=35pt]{../livrables/img/new/database.png}} ;
		\node[opacity=\op] () at \pos {\includegraphics[width=30pt]{../livrables/img/docker-logo.png}} ;
	}
	
	\foreach \pos/\image in {{(2.5,-5.3)}/debian_logo, {(3.3,-5.3)}/Arch-linux-logo, {(4.1,-5.3)}/redhat_logo}
		\node[] () at \pos {\includegraphics[width=15pt]{../livrables/img/\image.png}} ;

	\node[] () at (2.8,-6.2) {\includegraphics[width=15pt]{../livrables/img/new/processor32.png}} ;
	\node[] () at (3.8,-6.2) {\includegraphics[width=15pt]{../livrables/img/new/processor64.png}} ;
	\node at (1.55,-6.3) [align=center, right, rotate=90] {\huge \ldots} ;
	\node at (5.05,-6.3) [align=center, right, rotate=90] {\huge \ldots} ;	
	
	\draw[-stealth,very thick] (J) -- (D1);
	\draw[-stealth,very thick, opacity=0.5] (J) -- (D2);
	\draw[-stealth,very thick, opacity=0.2] (J) -- (D3);
	\node at (0.1,-7.6) [align=center, right] {Compilation} ;
	\draw[dashed] (6.3,-6.8) -- (6.3,1) -- (0,1) -- (0,-6.8) -- cycle ;	

	\node[] (D4) at (5,0) {\includegraphics[width=35pt]{../livrables/img/new/database.png}} ;
	\node[] () at (5,0) {\includegraphics[width=30pt]{../livrables/img/docker-logo.png}} ;
	
	\node[opacity=0.5] (D5) at (5,-2) {\includegraphics[width=35pt]{../livrables/img/new/database.png}} ;
	\node[opacity=0.5] () at (5,-2) {\includegraphics[width=30pt]{../livrables/img/docker-logo.png}} ;
	\node[opacity=0.2] (D6) at (5,-4) {\includegraphics[width=35pt]{../livrables/img/new/database.png}} ;
	\node[opacity=0.2] () at (5,-4) {\includegraphics[width=30pt]{../livrables/img/docker-logo.png}} ;
	%\draw[dashed] (3.7,-5.3) -- (3.7,1) -- (6.3,1) -- (6.3,-5.3) -- cycle ;	
	\node at (3.7,-7.6) [align=center, right] {Tests \\ d'installation} ;
	
	\draw[-stealth,very thick] (D1) -- (D4);
	\draw[-stealth,very thick, opacity=0.5] (D2) -- (D5);
	\draw[-stealth,very thick, opacity=0.2] (D3) -- (D6);
	
	\node[] (M1) at (8,0) {\includegraphics[width=35pt]{../livrables/img/new/database.png}} ;
	\node[] () at (8,0) {\includegraphics[width=20pt]{../livrables/img/new/box_white.png}} ;
	
	\node[opacity=0.5] (M2) at (8,-2) {\includegraphics[width=35pt]{../livrables/img/new/database.png}} ;
	\node[opacity=0.5] () at (8,-2) {\includegraphics[width=20pt]{../livrables/img/new/box.png}} ;
	\node[opacity=0.2] (M3) at (8,-4) {\includegraphics[width=35pt]{../livrables/img/new/database.png}} node at (M3.south east) [below] {Miroirs (dépôts)} ;	
	\node[opacity=0.2] () at (8,-4) {\includegraphics[width=20pt]{../livrables/img/new/box.png}} ;	
	
	\draw[-stealth,very thick] (D4) -- (M1);
	\draw[-stealth,very thick, opacity=0.5] (D5) -- (M2);
	\draw[-stealth,very thick, opacity=0.2] (D6) -- (M3);	
	
\end{tikzpicture}
\end{frame}

\section{Avancement}
\subsection{Où en sommes nous ?}
\begin{frame}{Où en sommes nous ?}
Les décisions actées :
\begin{itemize}
	\item Finalisation du fichier de configuration
	\begin{itemize}
		\item Rappel : format YaML
	\end{itemize}	
	\item Développement du projet en PHP
	\begin{itemize}
		\item Découpage en fonctions (charger, normaliser, élaguer...)
	\end{itemize}
\end{itemize}
\end{frame}

\subsection{Structure}
\begin{frame}{Structure}
	\begin{figure}[h!]
		\begin{tikzpicture}
			\node () at (0, 0) {\includegraphics[width=35pt]{../livrables/img/latex_puzzle.png}} ;
		\end{tikzpicture}
	\end{figure}
	
Nos propositions :
	\begin{itemize}
		\item Fractionnement du logiciel en plusieurs fichiers
		\item Utilisation de modules PHP
		\item Exploitation sur les distributions Debian (Testing et Stable), Archlinux (Rolling Release) et Fedora
	\end{itemize}
\end{frame}

\subsection{Structure}
\begin{frame}{Structure}
	\begin{tikzpicture}
		\node[] (1) at (0,0) [align=center, right] {Charger YaML} ;
		\node[] (2) at (3.3,0) [align=center, right] {Vérifier} ;
		\node[] (3) at (5.6,0) [align=center, right] {Normaliser} ;
		\node[] (4) at (5.6,-1.5) [align=center, right] {Élaguer} ;
		\node[] (5) at (2.5,-1.5) [align=center, right] {Écrire YaML} ;
		\node[] (6) at (2.5,-3) [align=center, right] {{\huge Compilation}} ;
			
			
		\draw[-stealth,very thick] (1) -- (2);
		\draw[-stealth,very thick] (2) -- (3);
		\draw[very thick] (3) edge[out=360,in=0,->] (4);
		\draw[-stealth,very thick] (4) -- (5);
		\draw[very thick] (5) edge[out=180,in=180,->] (6);
	\end{tikzpicture}
\end{frame}

\end{document}